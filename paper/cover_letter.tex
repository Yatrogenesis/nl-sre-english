\documentclass[11pt]{letter}
\usepackage[utf8]{inputenc}
\usepackage[margin=1in]{geometry}
\usepackage{hyperref}

\signature{Francisco Molina-Burgos\\
Avermex Research Division\\
M\'erida, Yucat\'an, M\'exico\\
\href{mailto:fmolina@avermex.com}{fmolina@avermex.com}}

\address{Avermex Research Division\\
M\'erida, Yucat\'an, M\'exico}

\date{January 22, 2026}

\begin{document}

\begin{letter}{Editor-in-Chief\\
Computer Speech \& Language\\
Elsevier}

\opening{Dear Editor,}

I am pleased to submit the manuscript entitled \textbf{``NL-SRE-English: A Deterministic Semantic Disambiguation Engine for English''} for consideration for publication in \textit{Computer Speech \& Language}.

This work presents a pure Rust implementation of a deterministic semantic disambiguation engine specifically designed for English natural language processing. The key contributions include:

\begin{itemize}
    \item A comprehensive verb database with 1,514 entries supporting multi-category disambiguation across 25 functional categories
    \item BK-Tree optimized fuzzy search achieving 1.6$\times$ speedup over linear search
    \item Automatic contraction expansion for 50+ common English contractions
    \item Zero external dependencies, enabling WebAssembly and embedded deployment
    \item Sub-microsecond latency (0.08 $\mu$s for verb lookup) suitable for real-time applications
\end{itemize}

The deterministic nature of the engine provides complete interpretability and reproducibility, making it particularly suitable for safety-critical applications in robotics, industrial control systems, and AI assistants where transformer-based models may lack the required guarantees.

The implementation is publicly available on crates.io (\texttt{nl-sre-english}) and GitHub under the MIT license, facilitating reproducibility and community adoption.

This manuscript has not been published elsewhere and is not under consideration by any other journal. All authors have approved the manuscript and agree with its submission.

I believe this work aligns well with the scope of \textit{Computer Speech \& Language} given its focus on practical NLP implementations with demonstrated performance characteristics.

Thank you for considering this submission.

\closing{Sincerely,}

\end{letter}
\end{document}
